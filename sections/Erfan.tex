\section{Nanorobotic manipulation}
\subsection{2017 - A Vision-based Automated Manipulation System for the Pick-up of Carbon Nanotubes}
\begin{itemize}
\item A nanorobotic manipulation system allowing automated pick-up of carbon nanotube (CNT) based on visual feedback.
\item Histogram thresholding for automatic binarization... clearly distinguished CNTs from the substrate and other impurities under various image brightnesses and contrasts.
\item Segment detection method (SDM) to separate the CNT
and AFM cantilever during overlapping.
\item Delicate manipulation of CNTs is a large challenge because CNTs are generated in bulk.
\item Conventionally, the manipulation of nanomaterials is manually performed through teleoperation which is time consuming, highly skill dependent and unproductive. [1]
\item Visual feedback-based nanorobotic manipulation
provides an effective way to overcome these shortcomings. 
\item Great progress to visually detect the position
of such nanomaterials, such as [2] which uses the principle component analysis (PCA) to locate CNTs
\item Great efforts have been made to obtain the relative depth information from SEM such as [3]. The observation and manipulation of nano scale objects under SEM
\end{itemize}
1: \textit{2013 - Automated pickplace of silicon nanowires} \\
2: \textit{2009 - Nanolab: A nanorobotic system for automated pick-and-place handling and characterization of cnts} \\
3: \textit{2013 - A measurement method for micro 3d shape
based on grids-processing and stereovision technology}



Contact detection by visually tracking of the displacement of the free end of a nanowire. 


\section{Aerial manipulation}
\subsection{2017: Uncalibrated Visual Servo for Unmanned Aerial Manipulation}